\documentclass{article}

\usepackage{fancyhdr, graphicx, amssymb, caption, gensymb}
\usepackage[labelfont=bf]{caption}
\usepackage[top=1in, bottom=1in, left=1in, right=1in]{geometry}
\usepackage{setspace} \singlespacing %\onehalfspacing \singlespacing % %\doublespacing %\setstretch{1.1}
\DeclareGraphicsExtensions{.png}

\title{Week 3 Progress Report}
\author{Stephanie Kwan \\
\\
Mentors: Professor Mansi Kasliwal, Dr. Ryan Lau, Jacob Jencson
}
\begin{document}
\maketitle
\pagestyle{fancy}
\lhead{Stephanie Kwan}
\chead{Week 3 Progress Report}
\rhead{SURF 2016}

\section{Motivation and background}
In the field of time-domain astronomy, the primary objective of studying transient astronomical events (known simply as ``transients'') is to develop our understanding of how astronomical phenomena ranging from supernovae to interstellar medium originate and evolve over time. While stars and galaxies evolve on the order of millions and billions of years, transients typically occur on timescales of seconds to years. From a measurement standpoint, sites of potential or active transient activity are measured from time-to-time on Earth or spacecraft. Snapshots are taken in different parts of the electromagnetic spectrum (gamma-ray, X-ray, infrared, visible), which are divided further into well-defined filters with known sensitivities to incident radiation. 


The Spitzer InfraRed Intensive Transient Survey (SPIRITs) is an ongoing systematic search of 194 galaxies within 20 Mpc, on timescales ranging between a week and a year, to a depth of 20 magnitudes \cite{SPIRITS proposal}. The search is based on the Spitzer Space Telescope's Infrared Array Camera (IRAC), which ran out of coolant at one point and can currently only operate on 3.6 $\mu$m and 4.5$\mu$m channels. Whenever new measurements are available every few weeks, a pipeline automatically performs processing and image subtraction with archival data. Members of the SPIRITs group visually vet the candidates and perform follow-up studies on interesting transients and variables to determine whether they exhibit behavior that falls within known categories. The 10th cycle of SPIRITs has discovered over 40 IR transients and over 1200 IR variables, some of which are ``typical'' variables or explosive transients such as pulsating asymptotic giant branch stars or supernovae explosions respectively, but some are much more unique \cite{Email}. 


% \begin{figure}[h]
% 	\centering
% 	\includegraphics[width=0.6\textwidth]{PSR}
% 	\caption{(a) Schematic of the polyhedral specular reflector (PSR) design \cite{Eisler}. (a) The polyhedral specular reflector (PSR), with a hollow primary concentrator, a stack of seven tilted filters in a solid prism, the solid secondary concentrators, and their corresponding solar cells labeled with their bandgap energies. (b) The submodules in a module array, whose size will depend on factors such as the cost, weight, and power output characteristics of the submodules.}
% 	\label{PSR}
% \end{figure}

\section{Problem and approach}


\section{Progress and challenges}


\section{Goals for next month}


%==================================%
%| Bibliography 				   |
%==================================%
\begin{center}
\rule{450pt}{1pt}
\end{center}
\pagebreak
\begin{thebibliography}{9}
\bibitem{SPIRITS proposal}
	Kasliwal, Mansi, Yi Cao, Frank Masci, George Helou, Robert Williams, John Bally, Howard Bond, Patricia Whitelock, et al. \textit{SPIRITS: SPitzer InfraRed Intensive Transients Survey: General Proposal.} Carnegie Institution of Washington, n.d. Web. 

\bibitem{Email}
	Lau, Ryan. February 17th, 2016 email correspondance. 


\end{thebibliography}


\end{document}